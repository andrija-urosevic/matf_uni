\documentclass[12pt, a4paper, twocolumn]{article}
\usepackage[utf8]{inputenc}
\usepackage[T2A]{fontenc}
\usepackage[serbian]{babel}
\usepackage[margin=1in]{geometry} 
\usepackage{graphicx}
\usepackage{pgfplots}
\usepackage[backend=biber,
            natbib=true,
            url=false,
            doi=true,
            eprint=false
]{biblatex}

\pgfplotsset{compat=1.16}

\title{Capture The Flag (CTF) kao uvod u racunarsku bezbednost}
\author{Andrija Urošević\\Univerzitet u Beogradu\\Matematički fakultet}
\date{April, 2022}

\addbibresource{ctf_main.bib}

\begin{document}

\maketitle

\begin{abstract}
\end{abstract}

\section{Uvod}

\section{CTF takmičenja}

\section{Znanja i veštine koje se stiču kroz CTF}

Treniranje profesionalaca u oblasti računarske bezbednosti zahteva puno 
vremena i novca, ali pruža jedano veoma održivo globalno rešenje. Mnoge 
obrazovne institucije, društva informatičara, državne organizacije, i privatne
kompanije su svesne toga te konstantno uvode nove studijske programe, 
i kurseve. Jedan od tih studijskih programa je CSEC2017.

Pored formalnog obrazovanja, povećava se popularnost neformalnih metoda.
CTF predstavlja jednu takvu metodu gde učesnici poboljšavaju svoje znanje u
oblasti računarske bezbednosti kroz razne zadatke. Kako CTF zadaci često 
poseduju takmičarske elemente i elemente igre, oni su neformalnog karaktera
i teško je odrediti njihovu vezu sa formalnim metodama.

\subsection{CSEC2017 oblasti znanja}

CSEC2017 definiše osam oblasti znanja () u računarskij bezbednosti.
\begin{enumerate}
    \item \emph{Bezbednost podataka} sadrži kriptografiju, forenziku, 
        integritet podataka, i autentifikaciju.
    \item \emph{Bezbednost softvera} se fokusira na bezbednost u 
        programiranju, testiranju, i druge aspekte razvoja softvera.
    \item \emph{Bezbednost komponenti} se odnosi na bezbednosti komponenti 
        koje se integrišu u veće sisteme, sto uključuje njihov dizajn i 
        obrnuto inženjerstvo.
    \item \emph{Bezbednost konekcije} podrazumeva mrežne servise, odbrane, i 
        napade.
    \item \emph{Bezbednost sistema} sadrži kontrolu pristupa, i etičko 
        hakovanje.
    \item \emph{Bezbednost ljudi} se odnosi na čuvanju identiteta, podataka, 
        i privatnosti. Sadrži socialno inženjerstvo i svesnost o računarskoj 
        bezbednosti.
    \item \emph{Organizaciona bezbednost} se bavi menadžmentom rizika, 
        bezbednostnim polisama, i rukovanjem incidenata na nivou organizacije.
    \item \emph{Društvena bezbednost} se bavi računarskom bezbednošću 
        na nacionalnom ili globalnom nivou.
\end{enumerate}

\subsection{Distribucija oblasti znanja u CTF zadacima}

Švábenský i dr.\cite{ctf_skills} ispitivali su distribuciju oblasti znanja u 
CTF zadacima. Ispitivanje je vršeno nad podacima, koji čine $5963$ 
\emph{writeup}-ova. Ovi podaci su preuzeti sa \url{CTFTime.org} koji u svojoj 
bazi, između ostalog,čuva i \emph{writeup}-ove raznih zadataka sa CTF 
takmičenja.\cite{ctf_time}

\textbf{Metod.} Prva faza je izdvajanje ključnih reči iz CSEC2017, koje 
određuju svako od znanja. Druga faza je preuzimanje \emph{writeup}-ova sa 
\url{CTFTime.org}. Treća faza predstavlja analizu \emph{writeup}-ova, tj.\ 
brojenje frekvenci ključnih reči u njima. Sledeća, četvrta faza predstavlja, 
normalizaciju frekvenci ključnih reči. Poslednja, peta faza se sastoji u 
dodeljivanju \emph{writeup}-ova odgovarajućoj oblasti znanja.\cite{ctf_skills}

\textbf{Rezultati.} Najzastupljenija oblast znanja je 
\emph{bezbednost podataka}, dok \emph{bezbednost konekcije} i 
\emph{bezbednost sistema} zauzimaju drugo i treće mesto, respektivno. 
\emph{Bezbednost podataka} uključuje kriptografiju, i autentifikaciju, te ima 
smisla biti na prvom mestu zbog same prirode zadataka iz te oblasti. Naime 
takvi zadaci su laki dizajn i proveru. Ostali rezultati se nalaze na 
slici~\ref{fig:ctf_ka}.\cite{ctf_skills} Dobijeni rezultati odgovaraju 
razulatitma o odnosu oblasti znanja na master studijskim programima iz 
kurseva o računarskoj bezbednosti.\cite{oth_ka, ctf_skills}

\begin{figure}
\begin{center}
    \begin{tikzpicture}[scale=0.6]
    \begin{axis}[
            y=1cm, 
            xbar, 
            title={Distribucija oblasti znanja u CTF-u}, 
            symbolic y coords={Društvo, Organizacija, Ljudi, Sistemi, Konekcije, Komponente, Softver, Podaci},
            legend pos = south east, 
            nodes near coords, 
            xmax=50
        ]
        \addplot+ coordinates {(2.96,Društvo) (9.86,Organizacija) (8.23,Ljudi) (12.72,Sistemi) (19.66,Konekcije) (8.94,Komponente) (10.02,Softver) (27.61,Podaci)}; 
        \addlegendentry{Jeopardy}
        \addplot+ coordinates {(2.17,Društvo) (11.38,Organizacija) (9.18,Ljudi) (10.96,Sistemi) (32.68,Konekcije) (2.08,Komponente) (14.72,Softver) (16.83,Podaci)}; 
        \addlegendentry{Attack-Defence}
    \end{axis}
\end{tikzpicture}
\end{center}
\caption{Distribucija oblasti znanja u $15 879$ \emph{jeopardy} i $86$ \emph{attack-defense} \emph{writeup}-ova.}\label{fig:ctf_ka}
\end{figure}

\textbf{Diskusija.} Glavna limitacija ove analize je u malom skupu podataka nad kojima je analiza vršena, zajedno sa odbacivanjem polovine skupa podataka radi nemogućnosti u parsiranju.\cite{ctf_skills} Postavlja se, takođe, pitanje o 
pouzdanosti samih \emph{writeup}-ova, tj.\ o njihovoj povezanosti sa samim
CTF zadatkom. Pod pretpostavkom da \emph{writeup}-ove pišu entuzijasti i sami
dizajneri CTF zadataka, možemo pretpostaviti njihovu pouzdanost.

\section{Problemi u CTF modelu}



\section{CTF vežbe na časovima}

\section{Zaključak}

\nocite{*}

\printbibliography.

\end{document}
