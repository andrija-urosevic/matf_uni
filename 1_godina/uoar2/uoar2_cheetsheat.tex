\documentclass[12p,a4paper]{article}

\usepackage[serbian]{babel} 
\usepackage[T2A]{fontenc} 
\usepackage[utf8]{inputenc} 
\usepackage{amsthm}
\usepackage{multicol} 
\usepackage[margin=0.5in]{geometry} 
\usepackage{amsmath} 
\usepackage{amsfonts} 
\usepackage{enumerate} 
\usepackage{amssymb}
\usepackage{tikz}
\usepackage{booktabs} 
\usepackage{graphicx} 

\DeclareMathOperator{\Dom}{Dom} 
\DeclareMathOperator{\Ima}{Im} 
\DeclareMathOperator{\nzd}{NZD} 
\DeclareMathOperator{\nzs}{NZS} 
\DeclareMathOperator{\NOR}{NOR} 

\newtheorem*{theorem}{Teorema}
\newtheorem*{prop}{Tvrđenje}


\title{Uvod u Organizaciju i Arhitekturu Racunara 2 --- Cheat Sheet}
\author{Andrija Urošević}

\begin{document}

\maketitle

\begin{multicols}{2}

\section{Bulova Algebra}

    \subsection{Tabele Istinitosti}

    \begin{tabular}{*{8}{c}}
        A & B & AND & NAND & OR & NOR & NOT & XOR \\
        \midrule
        0 & 0 & 0 & 1 & 0 & 1 & 1 & 0 \\
        0 & 1 & 0 & 1 & 1 & 0 & 1 & 1 \\
        1 & 0 & 0 & 1 & 1 & 0 & 0 & 1 \\
        1 & 1 & 1 & 0 & 1 & 0 & 0 & 0 \\
    \end{tabular}

    \subsection{Osnovni Zakoni Bulove Algebre}

    \begin{itemize}
        \itemsep0em
        \item $x \cdot x = x, x + x = x$ (zakon idempotencije)
        \item $x \cdot 1 = x, x + 0 = x$ (zakon neutrala)
        \item $x \cdot (x + y) = x, x + x \cdot y$ (zakon apsorpcije)
        \item $\overline{\overline{x}} = x$ (zakon dvojne negacije)
        \item $\overline{x + y} = \overline{x} \cdot \overline{y},
            \overline{x \cdot y} = \overline{x} + \overline{y}$
            (De-Morganovi zakoni)
    \end{itemize}

    \subsection{Logicki Izrazi i Normalne Forme}

    Pridruzivanje 0 i 1 logickim promenljivima naziva se \emph{valuacija}, tj.
    \emph{valuacija} je bilo koje preslikavanje iz skupa promenljivih $P$ u 
    $\{0, 1\}$, $v: P \mapsto \{0, 1\}$. Ovakvih funkcija ima $2^{|P|}$

    Za dva logicka izraza $E_1, E_2$ kazemo da su \emph{ekvivalentni} ako imaju
    jednake vrednosti u svakoj valuaciji.

    \subsubsection{Konjuktivna i Disjunktivna Normalna Forma}

    \emph{Literal} je logicka izraz koji se sastoji od logickih promenljivih i 
    njihovig negacija ($x, \overline{y}, z$).
    
    \emph{Elementarna konjukcija} --- $\mathcal{EK}$ je logicki izraz koji se 
    sastoji od konjukcije literala ($x \overline{y} z$).

    \emph{Elementarna disjunkcija} --- $\mathcal{ED}$ je logicki izraz koji se
    sastoji od disjunkcija literala ($ x + \overline{y} + z$).

    \emph{Disjunktivna normalna forma} --- $\mathcal{DNF}$ se sastoji od 
    disjunkcija elementarnih konjukcija 
    ($x \overline{y} + \overline{x} y z + x z$).

    \emph{Konjuktivna normalna forma} --- $\mathcal{KNF}$ se sastoji od 
    konjukcija elementranih disjunkcija
    ($(x + \overline{y} ) \cdot (\overline{x} + y + z) \cdot (x + z)$).

    Algoritam svodjenja izraza $E$ na izraz $E'$ u $\mathcal{DNF}$:
    \begin{enumerate}
        \itemsep0em
        \item Eliminacija logickih konstanti 0 i 1.
        \item Eliminacija negacija na vise od jedne promenljive pomocu 
              De-Morganovih zakona.
        \item Primena distributivnosti.
    \end{enumerate}

    \subsubsection{Savrsena Konjuktivna i Disjunktivna Normalna Forma}

    Za $\mathcal{EK}$ kazemo da je \emph{savrsena} u odnosu na dati skup
    promenljivih $P$ ako sadrzi tacno po jedan literal za svako od 
    promenljivih iz $P$ ($P = \{x, y, z\}, x \overline{y} z$).

    Za $\mathcal{ED}$ kazemo da je \emph{savrsena} u odnosu na dati skup 
    promenljivih $P$ ako sadrzi tacno po jedan literal za svaku od
    prmenljivih iz $P$ ($P = \{x, y, z\}, x + \overline{y} + z$).

    Za $\mathcal{KNF}$ kazemo da je \emph{savrsen} ako su njegove 
    $\mathcal{ED}$ savrsene ($(x + y) (\overline{x} + y)$).

    Za $\mathcal{DNF}$ kazemo da je \emph{savrsen} ako su njegove 
    $\mathcal{EK}$ savrsene ($xy + \overline{x}y$).

    \subsection{Logicke Funkcije}
    
    \emph{Logicka funkcija reda $n$} je bilo koje preslikavanje
    $f: {\{0, 1\}}^n \mapsto \{0, 1\}$, koje svakoj $n$-torci logickih 
    vrednosti $(x_1, x_2, \ldots, x_n) \in {\{0, 1\}}^n$ pridruzuje vrednost
    $y = \{0, 1\}$, tj. $f(x_1, x_2, \ldots, x_n) = y$.

    Domen funkcije $f$ ima $2^n$ vrednosti, dok kodomen ima $2$ vrednosti. 
    Sledi da funkcija ima ukupna $2^{2^n}$ preslikavanja.

    Funkcije reda 1:

    \begin{tabular}{*{2}{c}}
        Ime funkcije        & Vrednost funkcije \\
        \midrule
        Nula funkcija       & $f(x) = 0$ \\
        Jedinicna funkcija  & $f(x) = 1$ \\
        Funkcija identiteta & $f(x) = x$ \\
        Funkcija negacije   & $f(x) = \overline{x}$ \\
    \end{tabular}

    Funkcije reda 2:

    \begin{tabular}{*{2}{c}}
        Ime funkcije                                    & Vrednost funkcije \\
        \midrule
        Nula funkcija                                   & $f(x, y) = 0$ \\
        Jedinicna funkcija                              & $f(x, y) = 1$ \\
        Prva projekcija                                 & $f(x, y) = x$ \\
        Druga projekcija                                & $f(x, y) = y$ \\
        Negacija prve projekcije                        & $f(x, y) = \overline{x}$ \\
        Negacija druge projekcije                       & $f(x, y) = \overline{y}$ \\
        Konjukcija                                      & $f(x, y) = x y$ \\
        Disjunkcija                                     & $f(x, y) = x + y$ \\
        Seferova funkcija (NAND)                        & $f(x, y) = \overline{xy} = \overline{x} + \overline{y}$ \\
        Pirsova funkcija (NOR)                          & $f(x, y) = \overline{x + y} = \overline{x} \ \overline{y}$ \\
        Implikacija $x \implies y$                      & $f(x, y) = \overline{x} + y$ \\
        Implikacija $y \implies x$                      & $f(x, y) = \overline{y} + x$ \\
        Negacija implikacija $\overline{x \implies y}$  & $f(x, y) = x \overline{y}$ \\
        Negacija implikacija $\overline{y \implies x}$  & $f(x, y) = \overline{x} y$ \\
        Ekskluzivna disjunkcija $y \oplus x$            & $f(x, y) = x \overline{y} + \overline{x} y$ \\
        Ekvivalencija                                   & $f(x, y) = x y + \overline{x} \ \overline{y}$ \\
    \end{tabular}

    \subsubsection{Savrsena Disjunktivna/Konjuktivna Normalna Forma}

    \begin{tabular}{*{4}{c}}
        $x$ & $y$ & $z$ & $f(x, y, z)$ \\
        \midrule
        0 & 0 & 0 & 1 \\
        0 & 0 & 1 & 0 \\
        0 & 1 & 0 & 0 \\
        0 & 1 & 1 & 1 \\
        1 & 0 & 0 & 0 \\
        1 & 0 & 1 & 1 \\
        1 & 1 & 0 & 0 \\
        1 & 1 & 1 & 0 \\
    \end{tabular}

    Postupak za formiranje savrsene $\mathcal{DNF}$:
    \begin{enumerate}
        \itemsep0em
        \item Za svaku kombinaciju ulaznih vrednosti koja je tacna, tj. 1, 
              formiramo savrsenu elementarnu konjukciju koja je tacna samo u 
              toj valuaciji od promenljivih $x, y, z$.
        \item Formiranje savrsene disjunktivne normalne forme od tako 
              dobijenih elementarnih konjukcija.
    \end{enumerate}
    Za gornju tabelu dobijamo: 
    $f(x, y, z) = \overline{xyz} + \overline{x}yz + x\overline{y}z$.

    Postupak za Formiranje savrsene $\mathcal{KNF}$:
    \begin{enumerate}
        \itemsep0em
        \item Za svaku kombinaciju ulaznih vrednosti koja je netacna, tj. 0,
              formiramo savrsenu elementarnu disjunkciju koja je netacna samo u
              toj valuaciji od promenljivih $x, y, z$.
        \item Formiranje savrsene konjuktivne normalne forme od tako 
              dobijenih elementranih disjunkcija.
    \end{enumerate}
    Za gornju tabelu dobijamo:
    $f(x, y, z) = (x + y + \overline{z}) 
                  (x + \overline{y} + z)
                  (\overline{x} + y + z)
                  (\overline{x} + \overline{y} + z)
                  (\overline{x} + \overline{y} + \overline{z})$.

    \subsubsection{Potpuni Skupovi Veznika}

    \emph{Potpnuni skup veznika} je skup veznika pomocu koga se mogu izraziti
    sve ostale logicke funkcije

    Ako je skup veznika $C$ potpuni skup veznika, tada je i svaki njegov 
    nadksup $C$ takodje potpuni skup veznika.

    Minimalni potpnu skupovi veznika: $C^\cdot = \{\cdot, ^-\}$, 
    $C^+ = \{+, ^-\}$, $C^\uparrow = \{\uparrow\}$, 
    $C^\downarrow = \{\downarrow\}$.
    \[
        x \cdot y = 
        \overline{\overline{x \cdot y}} = 
        \overline{\overline{x \cdot y} \cdot \overline{x \cdot y}} =
        (x \uparrow y) \uparrow (x \uparrow y)
    \]
    \[
        x + y =
        \overline{\overline{x + y}} =
        \overline{\overline{x + y} + \overline{x + y}} =
        (x \downarrow y) \downarrow (x \downarrow y)
    \]

    
    \subsection{Minimizacija Logickih Izraza}

    \emph{Slozenost izraza} je broj beznika koje se pojavljuju u izrazu.

    \emph{Minimizacija} je pronalazenje logickog izraza minimalne slozenosti
    koji izracunava neku funkciju zadatu tabelarno.

    Minimizacija je znacajna u procesu dizajna logickih kola koja u savremenim 
    racunarima implementiraju logicke izraze, zbog ustede u procesu 
    proizvodnje i potrosnje elektricne energije.

    \subsubsection{Metod algebarskih transformacija}
    
    \begin{enumerate}
        \itemsep0em
        \item Ako izraz sadrzi dve elementarne konjukcija oblika $xK$ i 
              $\overline{x}K$, gde je $K$ proizvoljna konjukcija literala, tada
              imamo $xK + \overline{x}K = (x + \overline{x})K = K$
        \item Ukoliko jednu istu konjukciju $K$ mozemo grupisati na dva nacina
              sa $K_1$ i $K_2$, tada se primenom zakona idempotencije 
              ($K = K + K$), konjukcija moze grupisati i sa $K_1$ i sa $K_2$.
    \end{enumerate}

    Primer:

    \begin{align*}
        F(x, y, z) &=
        \overline{xyz} + \overline{xy}z + \overline{x}yz + 
        x\overline{y}z + \overline{x}y\overline{z} \\
        &= \overline{xyz} + (\overline{xy}z + \overline{xy}z) + 
        \overline{x}yz + x\overline{y}z + \overline{x}y\overline{z} \\
        &= (\overline{xyz} + \overline{xy}z) + 
        (\overline{x}y\overline{z} + \overline{x}yz) + 
        (\overline{xy}z +  x\overline{y}z) \\
        &= \overline{xy} + \overline{x}y + \overline{y}z
    \end{align*}

    \subsubsection{Metod Karnoovih mapa}

    \emph{Karnoova mapa} ja tablica pravougaonog oblika ciji je ukupan broj 
    polja $2^n$, gde je $n$ broj promenljivih u $\mathcal{DNF}$ izrazu.
    Za $n = 3$ imamo pravougaonu tablicu dimenzje $2 \times 4$, dok za $n = 4$
    imamo tablicu dimenzije $4 \times 4$. Svako polje tablice odgovara jednoj 
    valuaciji.
    \includegraphics[width=\columnwidth]{Figures/torus.png}

    Na pocetku se upise u svako polje odgovarajuca vrednost funkcije.
    Ako se dve jedinice nalaze jedna pored druge to znaci da se one mogu 
    grupisati, jer se razlikuju samo na jednom mestu.
    Slicno i ako imamo cetiri jedinice koje formiraju provougaonik
    Pravila zaokruzivanja:
    \begin{itemize}
        \itemsep0em
        \item Zaokruzuju se samo jedinice.
        \item Svaka jedinica mora da bude zaokruzena bar jednom.
        \item Mogu se zaokruzivati iskljucivo grupo od po $2^k$ polja 
              pravougaonog oblika.
        \item Uvek se zaokruzuju sto vece grupe, cak i ako se tom prilikom 
              neke jedinice ponovo zaokruzuju.
        \item Nakon sto se sve jednice zaokruze, treba proveriti da li je neko 
              od zaokruzivanja postalo suvisno, jer scako njegovo polje 
              pripada i nekom drugom zaokruzivanju.
    \end{itemize}

    Svakom od dobijenih zaokruzivanja odgovara jedna elementarna konjukcija
    koja sadrzi upravo one literale koju su zajednici za sva polja koja 
    obuhvata to zaokruzivanje.

    \subsection{Metod Kvin-Meklaskog}
    
    \section{Logicka Kola}
    
    \emph{Logicko kolo} (eng. \emph{logic circuit}) je uredjaj koji 
    implementira neki skup logickih funkcija u datoj tehnologiji.

    \subsection{Vrednost visoke impedense}
    
    Nekada je moguce da izlaz logickog nema nikakvu vrednost, tj.\ da mu
    izlazna vrednost nije ni 0 ni 1. Tu vrednost nazivamo 
    \emph{vrednost visoke impedense} i obelezavamo je sa $\mathbf{Z}$. Ukoliko
    neki izraz ima vrednost $\mathbf{Z}$, tada taj izraz ne uticne na vrednost 
    ulazna na koji je povezan.

    \subsection{Logicke Kapije}

    \emph{Logicke kapije} ili \emph{gejtovi} su uredjaju koji implementiraju 
    logicke veznika iz izabranog skupa.

    \begin{tabular}{*{3}{c}}
        Naziv kola & Sematska oznaka \\
        \midrule
        buffer  & \includegraphics[width=30px]{Figures/buffer.png} \\
        NOT  & \includegraphics[width=30px]{Figures/not.png} \\
        AND  & \includegraphics[width=30px]{Figures/and.png} \\
        OR   & \includegraphics[width=30px]{Figures/or.png} \\
        NAND & \includegraphics[width=30px]{Figures/nand.png} \\
        NOR  & \includegraphics[width=30px]{Figures/nor.png} \\
        XOR  & \includegraphics[width=30px]{Figures/xor.png} \\
        XNOR & \includegraphics[width=30px]{Figures/xnor.png} \\
    \end{tabular}

    \subsection{Implementacija Logickih Kapija u Savremenim Racunarima}

    Savremeni racunari su zasnivani na jednoj posebnoj vrsti unipolarnih
    tranzistora --- \emph{MOS tranzistori} 
    (eng. \emph{Metal-Oxide-Semiconductor}).
    MOS predstavlja poluprovodnicku komponentu koja ima tri prikljucka:
    \emph{sors} (eng. \emph{source}), \emph{drejn} (eng \emph{drain}), i 
    \emph{gejt} (eng. \emph{gate}).
    MOS tranzistor funkcionise kao prekidac --- struja moze da protice od 
    sorsa ka drajnu po uslovom da se odgovarajuci napon dovede na gejt.
    Kroz gejt ne protice struja vec on stvara elektricno polje kroz koje
    ce struja da tece.
    Postoje dva tipa MOS tranzistora: NMOS i PMOS tranzistor.

    Kod NMOS tranzistora sopr mora biti prikljucen na negativan, a drejn na 
    pozitivan napon.
    Da bi doslo do provodjenja struje, potrebno je na gejt dovesti dovoljno 
    veliki pozitivan napon.
    Ukoliko je napon u zoni logicke nule, tada je tranzistor u potpunosti 
    zatvoren i ne provodi struju od sorsa ka drejnu, dok kada je u zoni 
    logicke jedinice od provodi struju.

    Kod PMOS tranzistora sors se prikljucuje na pozitivan, a drajn na negativan
    napon. 
    Da bi doslo do provodjenja struje, potrebno je na gejt dovesti negativan 
    napon.
    Ukoliko je napon u zoni logicke jedinice, tada je tranzistor u zatvoren i
    ne provodi struju, dok kada je u zoni logicke nule on provodi struju.

    \includegraphics[width=0.8\columnwidth]{Figures/mos.png}

    Ako se u izradi logickig kola koriste i NMOS i PMOS tranzistori, tada
    govorimo o CMOS tehnologiji (eng. \emph{Complementary MOS}).

    \subsubsection{NOT kolo}

    \includegraphics[width=0.7\columnwidth]{Figures/mos_not.png}

    \subsubsection{NAND i AND kolo}

    \includegraphics[width=0.8\columnwidth]{Figures/mos_nand_and.png}

    \subsubsection{NOR i OR kolo}

    \includegraphics[width=0.8\columnwidth]{Figures/mos_nor_or.png}

    \subsubsection{XOR kolo}

    \includegraphics[width=0.7\columnwidth]{Figures/mos_xor.png}

    \subsubsection{Buffer}

    \includegraphics[width=0.4\columnwidth]{Figures/mos_buffer.png}

    \subsubsection{Propusni Tranzistori i Prenosne Kapije}

    \emph{Propusni tranzistori} kontrolisu propustanje nekog signala od jedne 
    tacke kola ka drugoj u zavisnost od vrednosti nekog drugog signala.

    \emph{Propusna kapija} CMOS varijanta propusnih tranzistora.

    \subsection{Baferi sa Tri Stanja}

    \includegraphics[width=0.7\columnwidth]{Figures/mos_tristate_buffer.png}
    
    \section{Kombinatorna Kola}

    \emph{Kombinatorna kola} (eng. \emph{combinationl curcuit}) su kola ko 
    kojih su izlazni signali jednoznacno odredjeni trenutnim ulaznim signlima, 
    tako da predstavljaju jednu logicku funkciju.

    \subsection{Osnovna kombinatorna kola}

    \subsubsection{Multiplekser}

    \emph{Multiplekser} je kolo sa $2^n$ ulaza, jednim izlazom i $n$ 
    upravljackih jedinica pomocu kojih se bira jedna od ulaza koji ce se 
    pojaviti na izlazu.
    
    \paragraph{Primena multipeksera}:
    \begin{itemize}
        \itemsep0em
        \item Bira podatak koji se salje na magistralu
        \item Operacija koju ALU mora da izracuna
        \item Dekompozicija logickih funkcija na jendostavnije funkcije
    \end{itemize}
    \includegraphics[width=0.7\columnwidth]{Figures/mux.png}

    \begin{tabular}{*{2}{c}}
        $S$ & $f(\mathbf{X}, S)$ \\
        \midrule
        00  & $X_1$ \\
        01  & $X_2$ \\
        10  & $X_3$ \\
        11  & $X_4$ \\
    \end{tabular}

    \subsubsection{Demultiplekser}

    \emph{Demultiplekser} je kolo sa jednim ulazom, $2^n$ izlaza i $n$
    upravljackih jedinica pomocu kojih se ulaz preusmerava na tacno jedan
    izlaz.

    \paragraph{Primena demultipleksera}
    \begin{itemize}
        \itemsep0em
        \item Bira odrediste vrednosti koja se prenosi nekom magistralom
    \end{itemize}
    \includegraphics[width=0.6\columnwidth]{Figures/dmx.png}

    \begin{tabular}{*{5}{c}}
        $S$ & $Y_1$ & $Y_2$ & $Y_3$ & $Y_4$ \\
        \midrule
        00  & $x$ &  0  &  0  &  0  \\
        01  &  0  & $x$ &  0  &  0  \\
        10  &  0  &  0  & $x$ &  0  \\
        11  &  0  &  0  &  0  & $x$ \\
    \end{tabular}

    \subsubsection{Dekoder}

    \emph{Dekoder} je kolo koje kao ulazni podatak prihvata $n$-bitni broj, a
    zatim na osnovu njega bira samo jedan on $2^n$ izlaza koji postavlja na 
    vrednost logicke jedinice.

    \paragraph{Primena decodera}
    \begin{itemize}
        \itemsep0em
        \item Odredjuje vrednost broja koji je zadat binarno
        \item Biranje registra iz kog se uzima vrednost
        \item Obrada sistema prekida
    \end{itemize}
    \includegraphics[width=0.4\columnwidth]{Figures/decoder.png}

    \begin{tabular}{*{5}{c}}
        $S$ & $Y_1$ & $Y_2$ & $Y_3$ & $Y_4$ \\
        \midrule
        00 & 1 & 0 & 0 & 0 \\
        01 & 0 & 1 & 0 & 0 \\
        10 & 0 & 0 & 1 & 0 \\
        11 & 0 & 0 & 0 & 1 \\
    \end{tabular}

    \subsubsection{Enkoder}

    \emph{Enkoder} je kolo koje ime $2^n$ ulaza, a zatim na osnovu njih stvara
    $n$-bitni broj.

    \emph{Enkoder sa prijoritetom} sprecava gresu kada se pojave vise od jedne
    jedinice na ulazu, te on ima prioritet prve jedinice koja se pojavljuje.

    \paragraph{Primene encodera}
    \begin{itemize}
        \itemsep0em
        \item 
    \end{itemize}
    \includegraphics[width=0.6\columnwidth]{Figures/encoder.png}

    \includegraphics[width=0.8\columnwidth]{Figures/priority_encoder.png}

    \subsubsection{Komparatori}

    \emph{Komparatori} su kola koja porede dve ulazne vrednosti.
    
    \includegraphics[width=0.6\columnwidth]{Figures/cmp_eq.png}

    \subsection{Aritmeticko-Logicka kola}


    \subsubsection{Pomeraci}

    \includegraphics[width=\columnwidth]{Figures/shifter.png}

    \subsubsection{Sabiraci i Oduzimaci}

    \paragraph{Polusabirac}

    \begin{tabular}{*{5}{c}}
        $X$ & $Y$ & $S$ & $C$ \\
        \midrule
         0  &  0  &  0  &  0  \\
         0  &  1  &  1  &  0  \\
         1  &  0  &  1  &  0  \\
         1  &  1  &  0  &  1  \\
    \end{tabular}

    \includegraphics[width=0.5\columnwidth]{Figures/half_adder.png}

    \paragraph{Punisabiraci}

    \begin{tabular}{*{5}{c}}
        $X$ & $Y$ & $PC$ & $S$ & $C$ \\
        \midrule
         0  &  0 &  0  &  0  &  0  \\
         0  &  0 &  1  &  1  &  0  \\
         0  &  1 &  0  &  1  &  0  \\
         0  &  1 &  1  &  0  &  1  \\
         1  &  0 &  0  &  1  &  0  \\
         1  &  0 &  1  &  0  &  1  \\
         1  &  1 &  0  &  0  &  1  \\
         1  &  1 &  1  &  1  &  1  \\
    \end{tabular}

    \includegraphics[width=0.7\columnwidth]{Figures/full_adder.png}

    \paragraph{Visebitni sabiraci}
    
    \emph{Visebitni talasasti sabirac} (eng. \emph{Ripple Carry Adder}) je kolo
    koje je implementirano tako da se za svaki bit pomoci sabiraca racuna zbir 
    i prenos, koji se onda prenosi na sledeci sabirac. Tako ulancani sabiraci,
    sa po $2\Delta$ kasnjenja, imaju ukupno $n \times 2\Delta$, jer se za svaki
    prethodni ceka racunanje prenosa. Pr.\ 32-bitni sabirac ima $64\Delta$ 
    kasnjenja.

    \includegraphics[width=0.7\columnwidth]{Figures/riplle_carry_adder.png}

    \paragraph{Poluoduzimac}
    
    \begin{tabular}{*{4}{c}}
        $X$ & $Y$ & $S$ & $C$ \\
        \midrule
         0  &  0  &  0  &  0  \\
         0  &  1  &  1  &  1  \\
         1  &  0  &  1  &  0  \\
         1  &  1  &  0  &  0  \\
    \end{tabular}

    \includegraphics[width=0.5\columnwidth]{Figures/half_sub.png}

    \paragraph{Punioduzimaci}

    \begin{tabular}{*{5}{c}}
        $X$ & $Y$ & $PC$ & $S$ & $C$ \\
        \midrule
         0  &  0 &  0  &  0  &  0  \\
         0  &  0 &  1  &  1  &  1  \\
         0  &  1 &  0  &  1  &  1  \\
         0  &  1 &  1  &  0  &  1  \\
         1  &  0 &  0  &  1  &  0  \\
         1  &  0 &  1  &  0  &  0  \\
         1  &  1 &  0  &  0  &  0  \\
         1  &  1 &  1  &  1  &  1  \\
    
    \end{tabular}

    \includegraphics[width=0.7\columnwidth]{Figures/full_sub.png}

    \paragraph{Visebitni oduzimac}

    \emph{Visebitni talasasti oduzimac} (eng. \emph{Ripple Carry Subtractor}) 
    je kolo koje je implementirano tako da se za svaki bit pomoci oduzimaca 
    racuna razliku i prenos, koji se onda prenosi na sledeci oduzimac. Tako 
    ulancani oduzimaci, sa po $2\Delta$ kasnjenja, imaju ukupno 
    $n \times 2\Delta$, jer se za svaki prethodni ceka racunanje prenosa. Pr.\ 
    32-bitni oduzimac ima $64\Delta$ kasnjenja.

    \includegraphics[width=0.7\columnwidth]{Figures/riplle_carry_sub.png}

    \subsubsection{Aritmeticko-Logicka Jedinica}

    \emph{Aritmeticko-Logicka jedinica} 
    (eng. \emph{Arithmetic Logic Unit --- ALU}) je kolo koje izvrsava razlicite
    aritheticke i logicke operacije, u zavisnosti od zahteva korisnika.

    \includegraphics[width=0.6\columnwidth]{Figures/alu.png}

    \subsubsection{Programabilni logicki nizovi}

    \emph{Programabilni logicki nizovi} 
    (eng. \emph{Programmable Logic Array --- PLA}) su programabilna logicka 
    kola koja sluze za implementiranje bilo koje logicke funkcije 
    predstavljene u savrsenom $\mathcal{DNF}$-u.

    \includegraphics[width=0.8\columnwidth]{Figures/pla.png}

    \subsubsection{ROM}

    \begin{tabular}{*{4}{c}}
        $X_1$ & $X_2$ & $Y_1$ & $Y_2$ \\
        \midrule    
        0 & 0 & 0 & 1 \\
        0 & 1 & 1 & 1 \\
        1 & 0 & 1 & 0 \\
        1 & 1 & 0 & 0 \\
    \end{tabular}

    \includegraphics[width=0.5\columnwidth]{Figures/rom.png}

    \section{Sekvencijalna Kola}

    Kod \emph{kombinatornih kola} vrednost na izlazu 
    $\mathbf{Y} = (y_1, y_2, \ldots, y_n)$ u nekom trenutku $t$ vazi 
    iskljucivo od vrednosti na ulazu $\mathbf{X} = (x_1, x_2, \ldots, x_n)$, 
    pa se moze predsaviti kao
    \[
        \mathbf{Y} = F(\mathbf{X}),
    \]
    gde je $F$ neka vektorska logicka funkcija po $\mathbf{X}$.

    Pod \emph{stanjem} podrazumevamo niz bitova 
    $\mathbf{S} = (s_1, s_2, \ldots, s_n)$, koja se pamte unutar kola.
    Da bi se to stanje odrzavalo samo sebe mora da zavisi i od $\mathbf{X}$ i 
    od $\mathbf{S}$:
    \[
        \mathbf{S} = G(\mathbf{X}, \mathbf{S}),
    \]
    gde je $G$ neka vektorska logicka funkcija po $\mathbf{X}$ i $\mathbf{S}$.
    Ovo se realizuje \emph{povratnom spregom}.
    
    \emph{Stabilono stanje} je stanje koje se nece promeniti sve dok se ulaz
    $\mathbf{X}$ funkcije $G$ ne promeni. Dok je \emph{nestabilno stanje}, 
    stanje za koje $\mathbf{S}$ nije fiksirana tacka funkcije $G$.

    \begin{itemize}
        \itemsep0em
        \item Vreme koje je potrebno da se stanje $\mathbf{S}$ stabilizuje u
              stabilno stanje $\mathbf{S'}$ naziva se 
              \emph{vreme stabilizacije}.
        \item Kolo koje osciluje izmedju razlicitih stanja i ne uspeva da 
              dostigne stbilno stanje naziva se \emph{nestabilno}.
        \item Kako kolo koje osciluje izmedju 0 i 1, tj. $+0V$ i $+5V$, u tom
              procesu promene moze da se stabilizuje na $+2.5V$, to stanje
              se naziva \emph{metastabilnost}.
        \item Nekada za dati ulaz $\mathbf{X}$ kolo moze da ode u stabilno 
              stanje $\mathbf{S'}$ ili $\mathbf{S''}$, u zavisnosti od 
              nepredvidivih fizickih faktora. Ova pojava se naziva 
              \emph{nedeterministicnost}.
    \end{itemize}

    Za kolo kazemo da je \emph{stabilno}, ako za svaku vrednost $\mathbf{X}$ 
    na ulazi i za svako trenutno stanje $\mathbf{S}$ kolo prelazi u stabilno
    stanje $\mathbf{S'}$ koje je jedinstveno odredjeno i zavisi samo od 
    $\mathbf{X}$ i $\mathbf{S}$, tj.\ imamo:
    \[
        \mathbf{S'} = T(\mathbf{X}, \mathbf{S}).
    \]
    Funkcija $T$ naziva se \emph{funkcija prelaska}.

    \subsection{Generatori radnog takta}

    \emph{Radni takt} ili \emph{casovnik} (eng. \emph{clock}) neprekidno 
    emituje impusle odredjene duzine u odredjenim vremenskim razmacima. 

    Prelazak sa 0 na 1 se naziva \emph{ulazna ivica} (eng. \emph{rising edge}).
    Prelazak sa 1 na 0 se naziva \emph{silazna ivica} 
    (eng. \emph{falling edge}).

    Vreme intervala izmedju dve odgovarajuce ivice dva uzastona impulsa naziva 
    se \emph{ciklus radnog takta} (eng. \emph{clock cycle time}).

    Broj ciklusa u sekundi odredjuju \emph{frekvenciju radnog takta}.

    \emph{Sinhroni} radni takt ima jednako trajanje ciklusa, Dok 
    \emph{asinhroni} nema.

    \subsection{Asinhrona i Sinhrona sekvencijalna kola}

    \emph{Asinhrona kola} su sekvencijalna kola kod kojih se menja stanje na
    osnovu izlaza nekog drugog kola.

    \emph{Sinhrona kola} su sekvencijalna kola koja menjaju stanje samo u 
    odredjenom trenutku ciklusa casovnika (uzlazni ili silazni signla).

    Kod sinhronih kola stanje se menja na odredjeno vreme sto je diskretna
    velicina, dok kod asinhronih kola vreme je kontinualna velicina.
    Da bi sinhrona kola radila ciklus radnog takta mora da bude duzi od 
    najveceg kasnjenja nekog kola, sto smanjuje brzinu kola koje imaju malo
    kasnjenje.

    \subsection{Reze}

    \subsubsection{SR reza}

    \emph{SR reza} (eng. \emph{SR latch}) je sekvencijalno kolo koje ima dva
    ulazna bita $S$ (set) i $R$ (reset) i dvobitno stanje $Q\overline{Q}$ koje 
    je ujedno i izlaz i definisano je sa:

    \begin{align*}
        Q &= R \ \NOR\ \overline{Q} \\
        \overline{Q} &= S \ \NOR\ Q 
    \end{align*}.
    \begin{itemize}
        \itemsep0em
        \item $S = 0, R = 0$ reza cuva prethodno stanje
        \item $S = 0, R = 1$ reza resetuje svoje stanje na 0
        \item $S = 1, R = 0$ reza setuje svoje stanje na 1
        \item $S = 1, R = 1$ je nestabilno stanje 
    \end{itemize}

    \begin{tabular}{*{5}{c}}
        $S$ & $R$ & $Q$ & $Q_n$ & Action \\
        \midrule
         0  &  0  &  0  &  0  & hold state\\
         0  &  0  &  1  &  1  & hold state\\
         0  &  1  &  x  &  0  & reset \\
         1  &  0  &  x  &  1  & set \\
         1  &  1  &  x  & --- & not allowed \\
    \end{tabular}

    \includegraphics[width=0.5\columnwidth]{Figures/sr_latch.png}

    \subsubsection{D reza}

    \emph{D reza} (eng. \emph{D lathc}) je sekvencijalno kolo, koje za
    razliku od SR reze ima samo jedan ulazni bit $D$ i jedan bit za kontrolu 
    menjanja stanja $e$. Kod D reze nemamo mogucnost nestabilnog stanja, zato
    sto je na $S$ ulazu $D$, a na $R$ ulazu $\overline{D}$. Iz tog razloga
    nikada nije moguce cuvati stanje za bilo koje $D$, pa se uvodi bit za 
    kontrolu stanja $e$, koji menja stanje u zavisnost od $D$ kada je 1, dok
    kada je 0 on cuva stanje. Ukratko vrednost sa $D$ se smesta u D rezu.

    \begin{itemize}
        \itemsep0em
        \item $e = 0$ stanje ostaje isto
        \item $e = 1, D = 0$ stanje se resetuje na 0
        \item $e = 1, D = 1$ stanje se setuje na 1
    \end{itemize}

    \begin{tabular}{*{5}{c}}
        $D$ & $Q$ & $e$ & $Q_n$ & Action \\
        \midrule
         x  &  0  &  0  &  0  & hold state\\
         x  &  1  &  0  &  1  & hold state\\
         0  &  x  &  1  &  0  & reset \\
         1  &  x  &  1  &  1  & set \\
    \end{tabular}

    \includegraphics[width=0.7\columnwidth]{Figures/d_latch.png}

    \subsection{Flip-Flop}

    Reze predstavljaju asihnrona sekvencijalna kola. Dok \emph{Flip-Flopovi} 
    predstavljaju najjednostavnija sinhrona kola. Za rad flip-flopova potreban
    je radni takt pa tako oni menjaju svoje stanje samo na ulaznoj/silaznoj
    ivici casovnika.

    \subsubsection{Master-Slave SR flip-flop}

    \includegraphics[width=0.8\columnwidth]{Figures/sr_ms_flipflop.png}

    Dok god je signal radnog takta 0, onda se stanje na prvoj SR rezi menja 
    ili ostaje isto u zavisnosti od $S$ i $R$. Kada se signal radnog takta
    promeni na 1, na ulaznoj ivici stanje se menja na drugoj SR rezi, koje
    ujedno i predstavlja stanje SR flip-flopa. Ako se dovedu dve jedinice na
    ulazu dolazi do nestabilnog stanja.

    \subsubsection{Master-Slave D flip-flop}

    \includegraphics[width=0.9\columnwidth]{Figures/d_ms_flipflop.png}

    Ako je enable bit $e = 0$, stanje se cuva, dok kada je enable bit 
    $e = 1$ stanje prve SR reze postavlja se na vrednost ulaza $D$, kada se
    signal radnog takta promeni sa 0 na 1, stanje se prenosi na drugu SR rezu,
    koje je ujedno i stanje D flip-flopa.

    \subsubsection{Master-Slave JK filp-flop}

    \includegraphics[width=0.9\columnwidth]{Figures/jk_ms_flipflop.png}

    Prva SR reza zavisi od trenutnog stanja JK flip-flopa i ulaza $J, K$. Na
    ulaznoj ivici casovnika stanje sa prve SR reze prebacuje se u drugu sto je
    ujedno i stanje JK flip-flopa.

    \begin{itemize}
        \itemsep0em
        \item $J = 0, K = 0$ stanje ostaje isto
        \item $J = 0, K = 1$ stanje se resetuje na 0
        \item $J = 1, K = 0$ stanje se setuje na 1
        \item $J = 1, K = 1$ stanje prelazi u svoj komplement
    \end{itemize}

    \subsubsection{Master-Slave T flip-flop}

    \includegraphics[width=0.9\columnwidth]{Figures/t_ms_flipflop.png}

    Prva SR reza zavisi od trenutnog stanja T flip-flopa i ulaza $T$. Na
    ulaznoj ivica cosovnika stanje sa prve SR reze prebacuje se u drugu sto je
    ujedno i stanje T flip-flopa.

    \begin{itemize}
        \itemsep0em
        \item $T = 0$ stanje ostaje isto
        \item $T = 1$ stanje prelazi u svoj komplement
    \end{itemize}

    \subsubsection{1s Catching Problem}

    \emph{1s catching problem} nastaje kada se na $J$ ili $K$ dovede 1 i pre
    uzlazne ivice casovnika onda vrati na 0. Nakon otkucaja radnog takta 
    stanje JK flip-flopa ce se promeniti u zavisnosti od prethodnog stanja, 
    jer je prva SR reza `uhvatila' tu jedinicu i promenila svoje stanje. 
    Resava se dodavanjem jednog multipleksera koji na ulazu ima 
    $Q, 0, 1, \overline{Q}$, a kontrolni bit mu je 2bitni zapis $JK$.
    \includegraphics[width=\columnwidth]{Figures/jk_ms_flipflop_1s.png}

    \subsection{Registri}

    \emph{Registar} je kolo koje cuva jedan visebitni binarni broj. Registri
    su napravljeni od jednog niza D flip-flopova, najcesce 8, 16, itd., 
    signalima za citanje i upis.
    \includegraphics[width=\columnwidth]{Figures/reg_8bit.png}

    \emph{Pomeracki registri} imaju mogucnost pomeranja svog sadrzaja u levo
    ili u desno u svakom ciklusu radnog takta, u zavisnosti od kontrolnih
    signala.
    \includegraphics[width=\columnwidth]{Figures/shift_reg.png}

    \subsection{Memorija}
    
    \emph{Staticka memorija} se konstruise pomocu registra koji su 
    hijerarhijski konstruisani pomocu D flip-flopova. Ova memorija je veoma
    brza pa se zato koristi kao kes memorija drugog nivoa.
    \includegraphics[width=\columnwidth]{Figures/mem28.png}
    \includegraphics[width=\columnwidth]{Figures/mem416.png}

    \subsection{Brojaci}

    \emph{Asinhroni brojac} radi tako sto stanje na svakog prethodnog kola
    utice na trenutno stanje kolo i tako se prenosi kao talas.

    \includegraphics[width=0.8\columnwidth]{Figures/asinhroni.png}

    \emph{Sinhroni brojac} radi tako sto stanje na svim kolima utice od 
    predhodnih stanja i menja se na svakom otkucaju casovnika.

    \includegraphics[width=0.9\columnwidth]{Figures/sinhroni.png}

    \subsection{Automat i Transdruktor}

    \section{Procesorska Organizacija i Performanse}

    \subsection{Datapath}
        
    \emph{Putanja podataka} (eng. \emph{datapath}) je skup funkcionalnih 
    jedinica kao sto je aritmeticko-logicka jedinica, registra, i magistrala.

    \includegraphics[width=\columnwidth]{Figures/1bus_datapath.jpg}
    \includegraphics[width=\columnwidth]{Figures/2bus_datapath.jpg}
    \includegraphics[width=\columnwidth]{Figures/3bus_datapath.jpg}
    
    
    

\end{multicols}

\end{document}
