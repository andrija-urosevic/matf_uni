\documentclass[12p, a4paper]{article}

\usepackage[serbian]{babel} 
\usepackage[T2A]{fontenc} 
\usepackage[utf8]{inputenc} 
\usepackage{amsthm}
\usepackage{multicol} 
\usepackage[margin=0.5in]{geometry} 
\usepackage{amsmath} 
\usepackage{amsfonts} 
\usepackage{enumerate} 
\usepackage{amssymb}
\usepackage{tikz}

\DeclareMathOperator{\Dom}{Dom} 
\DeclareMathOperator{\Ima}{Im} 
\DeclareMathOperator{\nzd}{NZD} 
\DeclareMathOperator{\nzs}{NZS} 

\newtheorem*{theorem}{Teorema}
\newtheorem*{prop}{Tvrđenje}
\newtheorem*{lema}{Lema}


\title{Diskretne Strukture 2}
\author{Andrija Urošević}

\begin{document}

\maketitle

\begin{multicols}{2}

\section{Kombinatorika}

\subsection{Prebrojavanje}

    \emph{Prebrojavanje} je odredjivanje kardinalnosti nekog skupa.

    \emph{Princim jednakosti}: Ako za skupove $A$ i $B$ postoji bijektivno 
    preslikavanje $f: A \mapsto B$ tada vazi:
    \[ |A| = |B|. \]
    
    \emph{Princim zbira}: Ako su $A_1, A_2, \ldots, A_n$ konacni disjunktni 
    skupovi, tj.\ $A_i \cap A_j = \emptyset, \forall i \neq j$, tada vazi:
    \[ 
        |A_1 \cup A_2 \cup \cdots \cup A_n| = 
        |A_1| + |A_2| + \cdots + |A_n| =
        \sum_{i=1}^n |A_i|.
    \]

    \emph{Princim proizvoda}: Neka su $A_1, A_2, \ldots, A_n$ konacni skupovi
    i neka je $A_1 \times A_2 \times \cdots \times A_n = 
    \{(a_1, a_2, \ldots a_n) | a_i \in A_i, i = 1,\ldots, n\}$. Tada vazi:
    \[
        |A_1 \cap A_2 \cap \cdots \cap A_n| = 
        |A_1| \cdot |A_2| \cdots |A_n| = 
        \prod_{i=1}^n |A_i|
    \]

    \emph{Dirihleov princip}: Ako je $n+1$ objekat smesteno u $n$ kutija, 
    postoji bar jedna kutija koja sadrzi bar $2$ objekta.

    \emph{Uopsteni Dirihlevo princip}: Ako je $k$ objekata smesteno u $n$ 
    kutija, pri cemu vazi $k > cn$, za $c \in \mathbb{N}$, tada postoji kutija 
    u kojoj je smesteno bar $c + 1$ objekat.

\subsection{Binomni koeficijent}

    \emph{Binomni keoficijent}, u oznaci $\binom{n}{k}$, definise se 
    $\forall k \in \mathbb{N}$ i $\forall n \in \mathbb{R}$ kao:
    \[ 
        \binom{n}{k} = 
        \frac{n (n - 1) \cdots (n - k + 1)}{k!}
    \]
    Dodatno $\binom{n}{k} = 1$. U specijalnom slucaju za 
    $n \in \mathbb{N}, k \leq n$, vazi jednakost
    \[
        \binom{n}{k} =
        \frac{n!}{k! (n - k)!}
    \]

    \subsubsection{Osnovna svojstva}

    \begin{enumerate}
        \itemsep0em
        \item [S.1] $\binom{n}{k} = \binom{n}{n - k}$ 
                    (\emph{Uslov simetricnosti})
        \item [S.2] $\binom{-n}{k} = {(-1)}^k \binom{n + k - 1}{k}$
                    (\emph{Negacija gornjeg indeksa})
        \item [S.3] $\binom{n}{k} = \binom{n - 1}{k} + \binom{n - 1}{k - 1}$
                    (\emph{Adiciona formula})
    \end{enumerate}
    
    \emph{Paskalov trougao}:

    \begin{tabular}{rccccccccc}
        $n=0$:&    &    &    &    &  1\\
            \noalign{\smallskip\smallskip}
        $n=1$:&    &    &    &  1 &    &  1\\
            \noalign{\smallskip\smallskip}
        $n=2$:&    &    &  1 &    &  2 &    &  1\\
            \noalign{\smallskip\smallskip}
        $n=3$:&    &  1 &    &  3 &    &  3 &    &  1\\
            \noalign{\smallskip\smallskip}
        $n=4$:&  1 &    &  4 &    &  6 &    &  4 &    &  1\\
            \noalign{\smallskip\smallskip}
    \end{tabular}

    \begin{theorem}[Binomna formula]
        Za $n \in \mathbb{N}$ i za 
        $x, y \in \mathbb{R}$ vazi:
        \[
            {(x + y)}^n = \sum_{k = 0}^n \binom{n}{k} x^k y^{n - k}.
        \]
    \end{theorem}

    \begin{theorem}
        Za $n \in \mathbb{N}$ i $x_1, x_2, \ldots, x_r \in \mathbb{R}$ vazi:
        \[
            {(x_1 + x_2 + \cdots + x_r)}^n = 
            \sum_{\substack{k_1 + k_2 + \cdots + k_r = n \\ 
                  k1, k2, \ldots, k_r \geq 0}}
            \binom{n}{k_1, k_2, \ldots, k_r} 
            x_1^{k_1} x_2^{k_2} \cdots x_r^{k_r}
        \]
    \end{theorem}

    \subsubsection{Binomni identiteti}

    \begin{enumerate}
        \itemsep0em
        \item [I.1] $\binom{n}{k} = \frac{n}{k} \binom{n - 1}{k - 1} = 
            \frac{n}{n - k} \binom{n - 1}{k}$
            (\emph{Izvlacenje ispred zagrade})
        \item [I.2] $\binom{n}{k + r} \binom{k + r}{k} = 
            \binom{n}{k} \binom{n - k}{k}$
            (\emph{Pojednostavljivanje proizvoda})
        \item [I.3] $\sum_{k = 0}^n \binom{r + k}{k} = \binom{n + r + 1}{n}$ i
            $\sum_{k = 0}^n \binom{k}{r} = \binom{n + 1}{r + 1}$
            (\emph{Sumacione formule})
        \item [I.4] $\sum_{k = 0}^n {\binom{n}{k}}^2 = \binom{2n}{n}$
            (\emph{Suma kvadrata})
    \end{enumerate}

    \subsubsection{Formula ukljucenja-iskljucenja}

    Za tri konacna skupa $A, B, C$ vazi jednakost:
    \[
        |A \cup B \cup C| = |A| + |B| + |C| - 
        (|A \cap B| + |A \cap C| + |B \cap C|) +
        |A \cap B \cap C|.
    \]
    
    \begin{theorem}[Formula ukljucenja-iskljucenja]
        Za konacne skupove $A_1, A_2, \ldots, A_n$ vazi jednakost:
        \[
            \left| \bigcup_{i = 1}^n A_i \right| = 
            \sum_{\emptyset \neq I \subset \mathbb{N}_n}
            {(-1)}^{|I| - 1} \left| \bigcap_{i \in I} A_i \right|
        \]
    \end{theorem}

    \subsection{Izbor elemenata}
    
    Neka je dat skup $S= \{a, b, c\}$. Biramo 2 iz njega:

    Uredjeni izbor elemenata:
    \begin{enumerate}
        \item Ponavljanje \textbf{je} dozvoljeno.
            \begin{tabular}{*{3}{c}}
                aa & ba &ca \\
                ab & bb &cb \\
                ac & bc &cc \\
            \end{tabular}
        \item Ponavljanje \textbf{nije} dozvoljeno.
            \begin{tabular}{*{3}{c}}
                ab & ba &ca \\
                ac & bc &cb \\
            \end{tabular}
    \end{enumerate}

    Ne uredjeni izbro elemenata:
    \begin{enumerate}
        \item Ponavljanje \textbf{je} dozvoljeno.
            \begin{tabular}{*{3}{c}}
                aa & bb & cc \\
                ab & ac & bc \\
            \end{tabular}
        \item Ponavljanje \textbf{nije} dozvoljeno.
            \begin{tabular}{*{3}{c}}
                ab & ac & bc \\
            \end{tabular}
    \end{enumerate}

    \subsubsection{Uredjeni izbor elemenata sa ponavljanjem --- Varijacije}

    Uredjeni izbor $k$ elemenata skupa od $n$ elemenata, uz dozvoljeno 
    ponavljanje prilikom izbora jednak je proju preslikavanja 
    $f: \mathbb{N}_k \mapsto \mathbb{N}_n$, tj.\
    \[
        n^k = \underbrace{n \cdot n \cdot \cdots \cdot n}_k.
    \]

    \subsubsection{Uredjeni izbor elemenata --- 
    Permutacije $\mathbf{k}$-tog reda}

    Uredjini izbor $k$ elemenata skupa od $n$ elemenata jednak je broju 
    injektivnih preslikavanja $f: \mathbb{N}_k \mapsto \mathbb{N}_n$, tj.
    \[
        \prod_{i = 0}^{k - 1} (n - i) = n(n - 1)(n - 2)\cdots(n - k + 1).
    \]

    \emph{Permutacije} su specijalan slucaj uredjenig izbora kod kojih vazi
    $k = n$. Sledi, broj permutacija skupa od $n$ elemenata jednak je
    \[
        n! = n(n - 1) \cdots 1.
    \]

    \subsubsection{Neuredjeni izbor elemenata --- Kombinacije}

    Broj neuredjenih izbora $k$ elemenata skupa od $n$ elemenata jednak je
    \[
        \binom{n}{k} = \frac{n!}{k! (n - k)!}.
    \]
    
    \subsubsection{Neuredjeni izbor elemenata sa ponavljanja}

    Broj neuredjenih izbora $k$ elementa skupa od $n$ elemenata, uz dozvoljeno 
    ponavljanje prilikom izbora, jednak je
    \[
        \binom{n + k - 1}{k} = \frac{(n + k - 1)!}{k!(n - 1)!}.
    \]

    Broj permutacija sa ponavljanjem multiskupa od $n$ elemenata u kojem se 
    prvi element pojavljuje $n_1$ puta, drugi element $n_2$ puta, $k$-ti  
    element $n_k$ puta, pri cemu bazi jednakost 
    $n = n_1 + n_2 + \cdots + n_k$, jednak je
    \[
        \binom{n}{n_1, n_2, \ldots, n_k} = \frac{n!}{n_1! n_2! \cdots n_3!}.
    \]

    \subsection{Funkcija generatrise}

    Svakom nizu $(a_n) = (a_0, a_1, \ldots, a_k, \ldots)$ mozemo jednoznacno 
    pridruditi stepeni red
    \[
        \sum_{i = 0}^\infty a_i x^i = a_0 + a_1 x + a_2 x^2 + \cdots 
        + a_k x^k + \cdots
    \]
    pri cemu je $x$ neka realna promenljiva. U slucaju da ovaj niz konvergira,
    njegovu sumu nazivamo \emph{funkcijom generatrise} niza $(a_n)$ i 
    oznacavamo je sa $a (= a(x))$.

    Nizu $(a_n) = (1, 1, \ldots, 1, \ldots)$ odogovara stepeni red 
    \[
        \sum_{i = 0}^\infty x^i = 1 + x + x^2 + \cdots + x^k + \cdots,
    \]
    te onda sledi da je funkcija generatrise jednaka niza $(a_n)$ jednaka
    \[
        a(x) = \sum_{i = 0}^\infty = \frac{1}{1 - x}.
    \]
    
    \begin{lema}
        Za $n \in \mathbb{N}$ i $x \in \mathbb{R}$ koeficijent koji se 
        nalazi uz $x^k$ u razvoju izraza ${(1 + x + x^2 + \ldots)}^n$ 
        jednak je
        \[
            \binom{n + k - 1}{k}.
        \]
    \end{lema}

    \begin{theorem}[Binomna formula za celobrojne eksponente]
        Za $n \in \mathbb{Z}$ i $|x| < 1$ vazi jednakost
        \[
            {(x + 1)}^n = \sum_{k = 0}^\infty \binom{n}{k} x^k.
        \]
    \end{theorem}

    \subsubsection{Odredjivanje funkcija generatrise}
    
    Neka su dati nizovi $(a_n)$ i $(b_n)$ i njihove funkcije generatrise 
    $a(x)$ i $b(x)$.

    \emph{Sabiranje nizova}: 
    \[
        (a_0 + b_0, a_1 + b_1, a_2 + b_2, \ldots) \leftrightarrow a(x) + b(x)
    \]

    \emph{Mnozenje niza skalarom}:
    \[
        (c a_0, c a_1, c a_2, \ldots) \leftrightarrow c a(x)
    \]

    \emph{Pomeranje niza za $k$ mesta udesno}:
    \[
        (\underbrace{0, 0, \ldots, 0}_k, a_0, a_1, a_2, \ldots) \leftrightarrow
        x^k a(x)
    \]

    \emph{Pomeranje niza za $k$ mesta ulevo}:
    \[
        (a_k, a_{k + 1}, a_{k + 2}, \ldots) \leftrightarrow 
        \frac{a(x) - (a_0 + a_1 x + a_2 x^2 + \cdots + a_{k - 1} x^{k - 1})}
        {x^k}
    \]

    \emph{Zamena $x$ sa $c x$}:
    \[
        (a_0, c a_1, c^2 a_2, \ldots, c^k a_k, \ldots) \leftrightarrow a(c x)
    \]

    \emph{Zamena $x$ sa $x^k$}: 
    \[
        (a_0, \underbrace{0, 0, \ldots, 0}_{k - 1}, a_1, 
        \underbrace{0, 0, \ldots, 0}_{k - 1}, a_2, \ldots) \leftrightarrow 
        a(x^k)
    \]

    \emph{Diferenciranje}:
    \[
        (a_1, 2 a_2, \ldots, k a_k, \ldots) \leftrightarrow a'(x)
    \]

    \emph{Integracija}:
    \[
        (0, a_0, \frac{1}{2} a_1, \frac{1}{3} a_2, \ldots, 
        \frac{1}{k + 1} a_k) \leftrightarrow \int a(x)\, \mathrm{d} x
    \]

    \section{Teorija brojeva}

    \subsection{Rekurentne jednacine}

    Za $n, k \in \mathbb{N}$ \emph{rekurentna jednacina $k$-tog reda} je oblika
    \[
        a_{n+k} = F (n, a_n, a_{n + 1}, \ldots, a_{n + k - 1}),
    \]
    gde su $a_n, a_{n + 1}, \ldots, a_{n + k} \in \mathbb{R}$ uzastopni 
    clanovi nekog niza.

    \subsubsection{Linearne rekurentne jednacine}

    Jednacinu oblika
    \[
        f_0(n) a_n + f_1(n) a_{n + 1} + \cdots + f_k(n) a_{n + k} = f(n)
    \]
    nazivamo \emph{linearna rekurentna jednacina $k$-tog reda}:
    \begin{itemize}
        \item \emph{sa konstantnim keoficijentima} $f_i = const$.
        \item \emph{sa funkcionalnim keoficijentima} $f_i = f_i(n)$
    \end{itemize}

    Za linearnu rekurentnu jednacinu kazemo da je homogena ako je $f(n) = 0$, 
    tj.\ ona je oblika:
    \[
        f_0(n) a_n + f_1(n) a_{n + 1} + \cdots + f_k(n) a_{n + k} = 0.
    \]

    Postupak resavanja homogene linearne rekurentne jednacine sa konstantnim
    keoficijentima:

    \begin{enumerate}
        \itemsep0em
        \item Postavimo karakteristicnu jednacinu \\
            $f_0 + f_1 x + f_2 x^2 +\cdots + f_k x^k = 0$.
        \item Ako su nule $x_1, x_2, \ldots, x_k$ nule karakteristicne 
            jednacine onda je opste resenje dato sa \\
            $a_n = c_1 x_1^n + c_2 x_2^n + \cdots c_k x_k^n$.
        \item Ako postoji nula $x_i$ visestrukosti $m$ karakteristicne
            jednacine onda je opste resenje dato sa \\
            $a_n = c_1 x_1^n + c_2 x_2^n + \cdots + 
            c_i x_i^n + c_{i + 1} n x_i^n + \cdots + 
            c_{i + m - 1} n^{m - 1} x_i^n + \cdots + c_k x_k^n$.
    \end{enumerate}
    
    Postupak resavanja nehomogene linearne rekurentne jednacine sa konstantnim
    keoficijentima:

    \begin{enumerate}
        \itemsep0em
        \item Resenje je oblika $a_n = h_n + p_n$, gde je $h_n$ opste resenje
            nehomogene linearne jednacine, a $p_n$ partikularno resenje date
            ne homogene linearne jednacine.
        \item Ako vazi $f(n) = P_d(n) b^n$, pri cemu je $P_d$ polinom stepena
            $d$, a $b$ relana konstanta, onda vazi sledece
        \begin{itemize}
            \itemsep0em
            \item ako je $x = b$ nije nula karakteristicne jednacine, onda
                partikularno resenje trazimo u \\
                $p_n = (A_0 + A_1 n + A_2 n^2 + \cdots + A_d n^d) b^n$;
            \item ako je $x = b$ jeste nula karakteristicne jednacine, onda
                partikularno resenje trazimo u \\
                $p_n = n^m(A_0 + A_1 n + A_2 n^2 + \cdots + A_d n^d) b^n$.
        \end{itemize}
    \end{enumerate}

    \section{Teorija grafova}
    

\end{multicols}

\end{document}
