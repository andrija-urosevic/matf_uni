\documentclass[12p,a4paper]{article}

\usepackage[serbian]{babel}
\usepackage[T2A]{fontenc} 
\usepackage[utf8]{inputenc} 
\usepackage{multicol} 
\usepackage[margin=0.5in]{geometry}
\usepackage{amsmath}
\usepackage{amsfonts} 

\DeclareMathOperator{\Ker}{Ker}
\DeclareMathOperator{\Ima}{Im}
\DeclareMathOperator{\Hom}{Hom}
\DeclareMathOperator{\GL}{GL}
\DeclareMathOperator{\sgn}{sgn}

\title{Linearna Algebra Cheat Sheet}
\author{Andrija Urosevic}

\begin{document}

\maketitle

\begin{multicols}{2}

\section{Uvod}

\subsection{Grupe. Prsten. Polja}

    Svaki uređeni par $(S, *)$ skupa $S$ i bilo koje od binarnih operacija $*$ 
    u tom skupu zovemo i jednim \textit{grupoidom}. Ako je ta operacija 
    asocijativna, tj važi $(a * b) * c = a * (b * c)$ kažemo da je taj grupoid 
    \textit{asocijativan}. Asocijativne grupide zovemo i \textit{polugrupama}.

    Za element $e \in M$ kažemo da je \textit{neutral} polugrupe $(M, *)$ ili 
    same operacije $*$, ako za svako $a \in M$ važi $a * e = e * a = a$. 
    Polugrupe sa neutralom zovemo i \textit{monoid}.

    Element $a$ je \textit{invertibilan} ako postoji bar jedno $a^- \in M$ za 
    koje važi $a * a^- = e$ i $a^- * a = e$.

    Monoid u kojima su svi elementi inverzibilni zovemo i \textit{grupom}. 
    Ako je ona i komutativna nazivamo je \textit{abelovom grupom}.

    Za preslikavanje $f : G \mapsto K$ kažemo da je \textit{homomorfizam} 
    grupoida $(G, *)$ u grupoid $(K, \circ)$ ako za svako $a, b \in G$ važi 
    $f(a * b) = f(a) \circ f(b)$

    Strukturu $(K, +, \cdot)$ nazivamo \textit{prstenom} ako je:
    \begin{enumerate}
        \itemsep0em
        \item $(K, +)$ je komutativna/Abelova grupa 
        \item $(K, \cdot)$ je monoid
        \item Operacija $\cdot$ je distrbutivna na operaciju $+$
    \end{enumerate}

    Strukturu $F = (F, +, \cdot)$ nazivamo \textit{polje} ako je:
    \begin{enumerate}
        \itemsep0em
        \item $(F, +)$ je Abelova grupa
        \item $(F \backslash \{0\}, \cdot)$ je Abelova grupa
        \item Operacija $\cdot$ je distributivana na operaciju $+$
    \end{enumerate}

    Polje racionalnih brojeva $\mathbb{Q} = \{ \frac{n}{m} | n \in \mathbb{Z} 
    \text{ i } m \in \mathbb{N}\}$.

    Polje realnih brojeva $\mathbb{R}$.

    Polje kompleksnih brojeva $\mathbb{C} = \{ a + ib | a,b \in \mathbb{R} \}$.

\section{Vektorski Prostori}
    
    \textit{Vektorski prostor nad nad poljem $F$} je uređena trojka 
    $V = (V, +, \cdot)$ koja sadrži skup $V$ i dve binarne operacije 
    (sabiranje i množenje skalarom respektivno)
    \[+ : V \times V \mapsto V, (x, y) \mapsto x + y\]
    \[\cdot : F \times V \mapsto V, (a, x) \mapsto ax\]
    tako da za svako $u,v \in V$ i svako $a, b \in F$ važi:
    \begin{enumerate}
    \itemsep0em
    \item [V.1] $(V, +)$ je Abelova grupa
    \item [V.2] $a(u + v) = au + av$
    \item [V.3] $(a + b)u = au + bu$
    \item [V.4] $a(bu) = (ab)u$
    \item [V.5] $1u = u$
    \end{enumerate}

\subsection{Vektorski potprostor}
    
    Neka je $V$ vektorski prostor nad poljem $F$ i neka je $V' \subset V$. 
    Ako je $V'$ vektorski prostor zatvoren za sabiranje i množenje skalarom 
    nad $V$, onda kažemo da je $V'$ \textit{potprostor vektorskog prostora} $V$

    Ako su $U$ i $W$ potprostori vektorskog prostora V, onda je 
    $U + W$ \textit{suma} vektorskih potprostora $U$ i $W$:
    \[U + W = \{ u + w | u \in U \text{ i } w \in W \}\]

    Kažemo da je suma \textit{direktran} i obeležavamo je sa $U \oplus W$, 
    ako važi da je
    \[U + W = V \text{  i  } U \cap W = \{0\}\]

\subsection{Linearna kombinacija}

    Neka je $V$ vektorski prostor nad poljem $F$ i neka su 
    $v_1, v_2, \ldots ,v_n$ vektori iz $V$. Onda kažemo da je $v \in V$ 
    \textit{linearna kombinacija vektora} $v_1, v_2, \ldots ,v_n$ ako postoje 
    $a_1, a_2, \ldots , a_n \in F$ takvi da:
    \[v = a_1 v_1 + a_2 v_2 + \cdots + a_n v_n\]

    Neka je $S$ podskup vektorskog prostora $V$. Skup svih linearnih 
    kombinacija vektora $v_1, v_2, \ldots ,v_n \in S$ naziva se 
    \textit{linearni omotač} vektorskog prostora $V$, tj. 
    $S$ \textit{razapinje} $V$, formalno:
    \[V = \mathcal{L} (S) = \{ \sum_{i=1}^n a_i v_i | 
    a_i \in F, v_i \in S, i = 1, 2 \ldots n \}\]

\subsection{Baza i Dimenzija}

    Neka je dat skup $S = \{ v_1, v_2, \ldots, v_n \}$ koji je podskup 
    vektorskog prostora $V$. Kažemo da su vektori iz $S$ 
    \textit{linearno nezavisni} ako jednačina:
    \[a_1 v_1 + a_2 v_2 + \cdots + a_n v_n = 0, 
    \\ a_i \in F, i = 1, 2, \ldots, n\]
    ima trivijalno rešenje, tj.\ da je $a_1 = a_2 = \cdots = a_n = 0$. 
    Vektori su \textit{linearno zavisni} ako nisu linearno nezavisni.

    Konačan skup vektora, $\{e_1, \ldots, e_n\}$ je \textit{baza} vektorskog 
    prostora $V$ ako je $\mathcal{L} (\{e_1, \ldots, e_n\}) = V$ i ako su 
    $e_1, \dots, e_n$ linearno nezavisni.

    Neka je $V$ vektorski prostor nad poljem $F$. I neka su $e_1, \ldots, e_n$ 
    baze tog vektorskog prostora, kažemo da je \textit{dimenzija} vektorskog 
    prostora kardinalni broj neke njegove baze, tj.\ $\dim U = n$.

    Neka su $U$ i $W$ neki potprostori vektorskog prostora $V$ dimenzije $n$. 
    Tada važi:
    \[\dim U + \dim W = \dim(U + W) + \dim(U \cap W)\]

\subsection{Rang}

    Neka su $v_1, v_2, \ldots, v_m$ konačnodimenzionalni sistem vektora nekog 
    vektorskog prostora $V$. Kažemo da je \textit{rang} sistema vektora 
    $r$, tj.\ $ rang (v_1, v_2, \ldots, v_m) = r$ ako važi:
    \begin{enumerate}
        \itemsep0em
        \item Postoji linearno nezavisan podskup od 
            $\{v_1, v_2, \ldots, v_m \}$ koji sadrži tacno $r$ vektora
        \item Svaki podskup od $\{v_1, v_2, \ldots, v_m \}$, koji se sadrži 
            od $r + 1$ vektor je linearno zavisan
    \end{enumerate}

\section{Linearno Preslikavanje}

    Neka su $U$ i $W$ vektorski prostori nad poljem $F$. Onda je preslikavanje 
    $L: U \mapsto W$ \textit{linearno}, ako za svako $u, v \in U$ i svako 
    $\lambda \in F$ važi:
    \[ L(u + v) = L(u) + L(v) \text{  i  } L(\lambda u) = \lambda L(u)\]

    Linearno preslikavanje $L : U \mapsto W$, 
    koje je injektivno zovemo \textit{monomorfizam}, 
    sujektivno preslikavanje zovemo \textit{epimorfizam}, 
    i bijektivno preslikavanje \textit{homomorfizam}. 
    Ako je $U = W$ onda kažemo da je linearno preslikavanje 
    \textit{endomorfno}, bijekcije ednomorfizma su \textit{automorfizmi}.

\subsection{Izomorfizmi Vektorskih Prostora}

    Neka su $U$ i $W$ dva vektorska prostora nad poljem $F$. Kažemo da je $U$ 
    \textit{izomorfno} na $W$ ako postoji izomorfizam $f : U \mapsto W$. 
    Ako su $U$ i $W$ izomorfni pišemo $U \cong W$

\subsection{Teorema o Rangu i Defektu}

    Neka su $U$ i $W$ dva vektorska prostora nad $F$ i neka je 
    $L : U \mapsto W$. Onda je \textit{jezgro} od $L$:\ 
    $\Ker L= \{v \in U | L(v) = 0 \}$, \textit{slika} od $L$:\ 
    $\Ima L = \{ L(v) | v \in U\}$, \textit{rang} linearnog preslikavanja je 
    $\rho (L) = \dim \Ima L$, i \textit{defekt} linearnog preslikavanja 
    $\delta(L) = \dim \Ker L$.

    Neka su $U$ i $W$ dva vektorska prostora nad nekim poljem $F$. 
    Pretpostavimo da je $U$ konačno-dimenzionalan i neka je 
    $L : U \mapsto W$ linearno preslikavanja. Onda je $\Ker L$ i $\Ima L$ 
    konačnodimenzionalni potprostori od $U$ i $W$, i imamo jednakost:
    \[\rho (L) + \delta (L) = \dim U \]

\subsection{Vektorski prostor $\Hom_F (U, W)$}

    Neka su $U$ i $W$ dva vektorska prostora nad poljem $F$. Uređena četvorka 
    $\Hom(U, W) = (\Hom_F(U, W), F, +, \cdot)$, gde je
    \[ (A + B) (x) = A(x) + B(x), \text{ za } A, B \in Hom(U, V), 
    \text { i } \forall x \in U\]
    \[ (\lambda A) (x) = \lambda A(x), \text{ za } A \in Hom(U, V), 
    \lambda \in F, \text { i } \forall x \in U\]
    vektorski porprostor nad poljem $F$ svih linearnih preslikavanja 
    iz $U$ u $W$.

\subsection{Opšta Linearna Grupa}

    Grupa svih inverzibilnih elemenata endomorfizma vektorskog prostora $V$ 
    nad poljem $F$ zove se \textit{opsta linearna grupa} vektorskog 
    prostora $V$ nad poljem $F$ i opise se $\GL_F(V)$. 
    Dodatno, $\GL_F(V) \cong \GL_n(F)$

    Neka je matrica $A \in \mathbb{M}_n (F)$, onda je $A$ \textit{regularna} 
    ako je inverzibilna, tj.\ $A \in \GL_n(F)$, ako nije inverzibilna onda 
    je \textit{singularna}.

    Skup matrica koje se razlikuju od jediničnih matrica samo na 
    jednom elementu nazivaju se \textit{elementarne matrice}.

\section{Determinante}

    Neka je dato jedinstveno linearno preslikavanje 
    $\det : \mathbb{M}_n (F) \mapsto F$, ako je $A \in \mathbb{M}_n(F)$ onda 
    $\det(A)$ nazivamo \textit{determinanta} matrice $A$. Determinante se 
    može zapisati kao:

    \[
        \det(A) = 
        \begin{vmatrix}
            a_{11} & \cdots & a_{1n} \\
            \vdots & &        \vdots \\
            a_{n1} & \cdots & a_{nn} 
        \end{vmatrix}
    \]
    \[ 
        \det(A) = \sum_{\sigma \in S_n} 
        \left( 
        \sgn(\sigma) \prod_{i=0} a_{i,\sigma(i)} 
        \right) 
    \]

\section{Skalarni proizvodi i Ortogonalnost}

    Neka je $V$ vektorski prostor nad poljem $F$. \textit{Skalarni proizvod}  
    nad $V$ je dodela koja paru elementata $v, w \in V$ dodeljuje skalar.
    Obelezava se sa $v \cdot w$ ili sa $<v, w>$ i zadovoljava svojstva:
    \begin{enumerate}
        \itemsep0em
        \item $v \cdot w$ = $w \cdot v$ za svaki $v, w \in V$
        \item $u \cdot (v + w) = u \cdot v + u \cdot w$, za $u,v,w \in V$
        \item $\lambda u \cdot v = \lambda (u \cdot v)$ i 
            $u \cdot \lambda v = \lambda (u \cdot v)$, za $\lambda \in F$
    \end{enumerate}
 
\end{multicols}


\end{document}
